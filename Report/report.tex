
\documentclass[conference]{acmsiggraph}

\usepackage{graphicx}
\graphicspath{{./images/}}

\newcommand{\figuremacroW}[4]{
	\begin{figure}[h] %[htbp]
		\centering
		\includegraphics[width=#4\columnwidth]{#1}
		\caption[#2]{\textbf{#2} - #3}
		\label{fig:#1}
	\end{figure}
}

\newcommand{\figuremacroF}[4]{
	\begin{figure*}[t] % [htbp]
		\centering
		\includegraphics[width=#4\textwidth]{#1}
		\caption[#2]{\textbf{#2} - #3}
		\label{fig:#1}
	\end{figure*}
}

\usepackage{afterpage}
\usepackage{xcolor}
\definecolor{lbcolor}{rgb}{0.98,0.98,0.98}
\usepackage{listings}

\lstset{
	escapeinside={/*@}{@*/},
	language=C++,
	%basicstyle=\small\sffamily,
	%basicstyle=\small\sffamily,	
	basicstyle=\fontsize{8.5}{12}\selectfont,
	%basicstyle=\small\ttfamily,
	%basicstyle=\scriptsize, % \footnotesize,
	%basicstyle=\footnotesize,
	%keywordstyle=\color{blue}\bfseries,
	%basicstyle= \listingsfont,
	numbers=left,
	numbersep=2pt,    
	xleftmargin=2pt,
	%numberstyle=\tiny,
	frame=tb,
	%frame=single,
	columns=fullflexible,
	showstringspaces=false,
	tabsize=4,
	keepspaces=true,
	showtabs=false,
	showspaces=false,
	%showstringspaces=true
	backgroundcolor=\color{lbcolor},
	morekeywords={inline,public,class,private,protected,struct},
	captionpos=t,
	lineskip=-0.4em,
	aboveskip=10pt,
	%belowskip=50pt,
	extendedchars=true,
	breaklines=true,
	prebreak = \raisebox{0ex}[0ex][0ex]{\ensuremath{\hookleftarrow}},
	keywordstyle=\color[rgb]{0,0,1},
	commentstyle=\color[rgb]{0.133,0.545,0.133},
	stringstyle=\color[rgb]{0.627,0.126,0.941}
}

\usepackage{lipsum}

\title{Optimising the LINPACK 1000 using Parallelization}

\author{Sam Serrels\\\ 40082367@napier.ac.uk \\
Edinburgh Napier University\\
Concurrent and Parallel Systems (SET10108)}
\pdfauthor{Sam Serrels}

\keywords{Vehicle Routing, Clarke Wright}

\begin{document}

\maketitle

\section{Introduction}

\paragraph{The LINPACK benchmark}
The Linpack algorithm is a popular benchmark in the high performance computing field as it is uses as a floating point performance measure for ranking supercomputers.
The Benchmark solves a large system of linear equations using LU decomposition and is typical of many matrix-based scientific computations.
Linpack started as a Fortran maths processing application, the code for solving linear equations was extracted from the original program and turned into a benchmark.  
\\
The Linpack1000 algorithm first generates a random 1000x1000 element matrix, A, and 1000 element vector, b. The elements in A and b are all double precision floating point numbers. 
Processing then takes place to find the solution, a 1000 element vector, x, such that Ax = b.
\\
This report documents the process of analysing a specific sequential Linpack implementation and then converting it to a parallel task.

\paragraph{Related Work}
As Linpack is used to benchmark large scale multiprocessor supercomputers, there are many parallel versions available. 
The main difference between implementations is the Gaussian elimination stage, for which there are many algorithms available, some of which can split the task up into separate easily parallelizable chunks of logic. For the scope of this project, changes to the algorithm were avoided wherever possible to keep a fair comparison to the original sequential code.

\paragraph{High-Performance Linpack}
The most commonly used implantation of Linpack is the HPL implementation, written by the Innovative Computing Laboratory at the University of Tennessee. 
HPL is an open-source project that aims to provide a toolbox for configuring, optimising, and running the benchmark over a network. 
It contains many version of the algorithm with plenty of configurable options and for tuning performance to a specific system.
\\
The HPL project was used as a rough guide to how the reference algorithm could be modified for this project, however most of the optimisation were beyond the scope of this project.

\paragraph{Project Scope}
A quick optimisation would be to drop the precision of the algorithm from double, to single precision floating point values.
Another method would be to swap the original Gaussian elimination algorithm for a different mathematical approach that would lend itself to parallel processing better.
This project aimed to see how much the original Linpack code can be optimised with parallelization, without changing the core logic of the algorithm or data output so these routes for optimisation were ruled out.

\paragraph{OpenMP}
The technology for processing the application in parallel was chosen to be OpenMp, an API that abstracts the creation of threads from the user and therefore allows for easier development and better cross platform portability, assuming that the chosen platform has a complier that supports OpenMP.
This was chosen over creating threads manually, mainly for ease of development reasons, but also because even in a situation that OpenMp is slower than Manual threads, there should still be a noticeable performance increase over the baseline results.

\paragraph{SIMD}
For an extra level of performance, SIMD instructions were used to gain performance in the most frequently executed parts of the program.

\section{Linpack Gaussian elimination}

\figuremacroW
{linpackmat}
{Overview of GE progression}
{}
{1.0}

\paragraph{Initial analysis}
\figuremacroW
{pie}
{Total program execution}
{}
{0.8}

The Gaussian elimination(GE) stage, work it's way through the array, starting in the "top left corner".
It examines the entire first column(C) ("Pivot and scale", figure \ref{linpackmat}) and finds the largest value (T).
The row that contains the largest value(T) becomes the pivot, it is swapped with the topmost row.
Then, each column is processed, by multiplying it by 1/T and then adding the value of the C column.

Once this is complete, the process restarts but in a subsection of the Matrix A that is one row and column smaller than the previous iteration.
This continues until the "Bottom right" corner is reached. This process transforms Matrix A into a upper triangular matrix that is in row echelon form.

On analysis of execution of the program , it was clear that the vast majority of the execution time was based in the Gaussian Elimination stage.
On further examination, the "Daxpy" function, which his called many times during the GE stage takes up nearly 90\% of the total execution time.
Daxpy is the function that does the scaling and addition of each column, and is called 424166 times during the full execution of the program.

\paragraph{Daxpy}
While the Daxpy function is called at a high frequency it contains very few lines of code.
It's function is to compute Y = S * X + Y, X and Y are elements of two arrays, and S is a scaler value.
In the program, this is used in a loop to process each column in the A array.
\\
This was the first part of the code to be examined for possible speed-up, as each iteration of the loop doesn't depend on any other iteration.
Initially Paralleling the loop with OpenMp was attempted, but as the loop will only ever execute a maximum of 1000 times,
 the overhead time of creating and running threads was always greater than the time of running the loop without threads.

\begin{lstlisting}[language=C++,caption={daxpy Code},label=daxpyCode]
void (int n,double scaler, double *dx, double *dy, int offset) {
	double *const y = &dy[offset];
	double *const x = &dx[offset];
	for (int i = 0; i < n; ++i) {
		y[i] += scaler * x[i];
	}
}
\end{lstlisting}

\begin{lstlisting}[language=C++,caption={Simd daxpy Code},label=SimdDaxpyCode]
void (int n, double scaler, double *dx, double *dy,int offset) {
	if ((n <= 0) || (scaler == 0)) {
		return;
	}
	double *const y = &dy[offset];
	double *const x = &dx[offset];
	
	const __m256d scalers = _mm256_set1_pd(scaler);
	const int remainder = n % 4;
	const int nm1 = n - 3;
	
	for (int i = 0; i < nm1; i += 4) {
		// load X
		const __m256d xs = _mm256_loadu_pd(&x[i]);
		// load y
		__m256d ys = _mm256_loadu_pd(&y[i]);
		// mutliply X by scalers, add to Y
		ys = _mm256_add_pd(ys, _mm256_mul_pd(xs, scalers));
		// load back into y
		_mm256_storeu_pd(&y[i], ys);
	}
	
	for (int i = n - remainder; i < n; ++i) {
		y[i] += scaler * x[i];
	}
}
\end{lstlisting}


\paragraph{The Pivot loop}



\paragraph{The Collumn loop}

\paragraph{Memory alignment}


\section{Optimisation Method}

\paragraph{Code Simplification}

\paragraph{Parallelised Daxpy}
 no route was over the capacity of the truck.

\paragraph{SIMD Daxpy}


\paragraph{MDaxpy}


\afterpage{\clearpage}


\clearpage
\begin{table}[b]
{
\resizebox{1.0\textwidth}{!}{
\begin{minipage}{\textwidth}
\centering
    \begin{tabular}{ccccccccc}
						    &	Allocate&	Create Input &	gaussian&	&	&	Total&	Total &	Speedup\\ 
		Name				&	Memory (ms)&	Numbers (ms)&	eliminate (ms)&	Solve (ms)&	Validate (ms)&	Time (ms)&	Speedup&	(With Simd)$^{*}$\\ \hline\hline \\
		Threads: 1 No Simd	&	0.48	&	5.22	&	157.27	&	0.69	&	5.27	&	168.93	&	0\%	&	0\%     \\
		Threads: 1 Simd128	&	0.45	&	5.00	&	152.68	&	0.67	&	5.06	&	163.85	&	3\%	&	0\%     \\
		Threads: 1 Simd256	&	0.44	&	5.06	&	142.04	&	0.67	&	5.14	&	153.34	&	9\%	&	0\%     \\
		Threads: 2 No Simd	&	0.42	&	5.48	&	78.76	&	0.68	&	5.52	&	90.86	&	46\%	&	46\%\\
		Threads: 2 Simd128	&	0.46	&	5.36	&	85.35	&	0.66	&	5.49	&	97.32	&	42\%	&	41\%\\
		Threads: 2 Simd256	&	0.45	&	4.82	&	72.86	&	0.63	&	4.83	&	83.59	&	51\%	&	45\%\\
		Threads: 3 No Simd	&	0.45	&	5.37	&	63.49	&	0.68	&	5.46	&	75.44	&	55\%	&	55\%\\
		Threads: 3 Simd128	&	0.46	&	5.52	&	60.00	&	0.66	&	5.64	&	72.28	&	57\%	&	56\%\\
		Threads: 3 Simd256	&	0.46	&	5.74	&	52.51	&	0.58	&	5.85	&	65.14	&	61\%	&	58\%\\
		Threads: 4 No Simd	&	0.44	&	5.28	&	42.67	&	0.62	&	5.33	&	54.34	&	68\%	&	68\%\\
		Threads: 4 Simd128	&	0.43	&	5.72	&	41.51	&	0.54	&	5.78	&	53.97	&	68\%	&	67\%\\
		Threads: 4 Simd256	&	0.48	&	5.80	&	39.61	&	0.55	&	5.91	&	52.36	&	69\%	&	66\%\\
		Threads: 5 No Simd	&	0.46	&	5.41	&	50.89	&	0.67	&	5.50	&	62.93	&	63\%	&	63\%\\
		Threads: 5 Simd128	&	0.45	&	5.20	&	49.49	&	0.67	&	5.24	&	61.05	&	64\%	&	63\%\\
		Threads: 5 Simd256	&	0.46	&	5.11	&	46.46	&	0.60	&	5.24	&	57.86	&	66\%	&	62\%\\
		Threads: 6 No Simd	&	0.50	&	5.80	&	44.44	&	0.75	&	5.93	&	57.41	&	66\%	&	66\%\\
		Threads: 6 Simd128	&	0.48	&	5.53	&	43.04	&	0.69	&	5.66	&	55.41	&	67\%	&	66\%\\
		Threads: 6 Simd256	&	0.49	&	5.55	&	41.63	&	0.65	&	5.72	&	54.04	&	68\%	&	65\%\\
		Threads: 7 No Simd	&	0.53	&	6.00	&	40.21	&	0.75	&	6.18	&	53.66	&	68\%	&	68\%\\
		Threads: 7 Simd128	&	0.52	&	5.40	&	40.02	&	0.73	&	5.55	&	52.22	&	69\%	&	68\%\\
		Threads: 7 Simd256	&	0.55	&	5.58	&	35.69	&	0.68	&	5.77	&	48.28	&	71\%	&	69\%\\
		Threads: 8 No Simd	&	0.51	&	5.01	&	43.25	&	0.80	&	5.17	&	54.75	&	68\%	&	68\%\\
		Threads: 8 Simd128	&	0.53	&	5.48	&	43.44	&	0.78	&	5.73	&	55.96	&	67\%	&	66\%\\
		Threads: 8 Simd256	&	0.51	&	5.47	&	38.64	&	0.71	&	5.72	&	51.05	&	70\%	&	67\%\\
		\hline
    \end{tabular}
   
    \caption[Table caption text]{Results of all tests\\
    	*Speed-up with Simd, compares times against the	Simd equivalent 1 Thread run}
    \label{table:name}
    \end{minipage} }
}
\end{table}

\figuremacroF
{graph2}
{Total Speed-up percentage, for each number of threads}
{}
{1.0}
\figuremacroF
{graph1}
{Total Speed-up (Relative to baseline Simd) percentage, for each number of threads}
{}
{1.0}
\clearpage

\figuremacroF
{t1i8simd256conc}
{Single Threaded, 6 runs, simd256 Daxpy}
{Overall system CPU utilisation}
{1.0}

\figuremacroF
{t1i8simd256cond}
{Single Threaded, 6 runs, simd256 Daxpy}
{Thread to CPU Core allocation}
{1.0}

\figuremacroF
{t1i8simd256conp}
{Single Threaded, 6 runs, simd256 Daxpy}
{Thread Status}
{1.0}

\figuremacroF
{t4i8simd256conc}
{4 Threads, 6 runs, simd256 Daxpy}
{Overall system CPU utilisation}
{1.0}

\figuremacroF
{t4i8simd256cond}
{4 Threads, 6 runs, simd256 Daxpy}
{Thread to CPU Core allocation}
{1.0}

\figuremacroF
{t4i8simd256conp}
{4 Threads, 6 runs, simd256 Daxpy}
{Thread Status}
{1.0}

\figuremacroF
{t8i8simd256conc}
{8 Threads, 6 runs, simd256 Daxpy}
{Overall system CPU utilisation}
{1.0}

\figuremacroF
{t8i8simd256cond}
{8 Threads, 6 runs, simd256 Daxpy}
{Thread to CPU Core allocation}
{1.0}

\figuremacroF
{t8i8simd256conp}
{8 Threads, 6 runs, simd256 Daxpy}
{Thread Status}
{1.0}


\section{Conclusions}
\paragraph{Computation time}
Both Implementations of the Clarke-Wright algorithms produced expected results, with the parallel version producing larger and fewer routes. As for the time taken to calculate, the performance is roughly equal.
The discrepancies shown in Figure \ref{fig:chart1} when the amount of customers increases beyond 800 is possibly due to optimisations carried out by the Java virtual machine. The total operations carried out is roughly the same for each algorithm, however the arrays are accessed and modified at different times, this is a possible cause for the difference in processing time.

\paragraph{Solution Quality}
The Parallel solution produced a large saving of up to a 600\% increase against the baseline cost, shown in Figure \ref{fig:chart2}. The Sequential solution produced a constant saving of around 200\%. These results are also shown in Figure \ref{fig:chart3}, showing the number of routes.

\paragraph{Edge Cases}
It is possible that a customer can be left out of all routes due to capacity constraints; this is checked for at the end of the calculation. If a customer is left over, it is seen if it would be possible to add it to any existing route and then if it would be more efficient than sending out a new truck. This can be seen in Figure \ref{fig:rand00020cwpsn}, the customer in the top left falls in to this edge case category.

\paragraph{Conclusion}
The implementation written for this report successfully computes optimised and usable data, the processing cost increases in a quadratic relation to the size of data. The specific implementation could be optimised to produce quicker results. One possible optimisation route could be a custom sort method, as profiling reported that 40\% of the processing time is taken by the initial sort of the customer pairs.\\
Overall this report produced repeatable and meaningful data, and can be seen as a successful investigation into the Clark-Wright Algorithm.
\bibliographystyle{acmsiggraph}
\bibliography{report}
\clearpage

\section{Appendix: Code}
\lstinputlisting[language=C++]{../src/cw1/sequentialOMP.cpp}

\end{document}

